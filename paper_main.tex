% GravExplain
\documentclass[prb,preprint]{revtex4-1} 
% \documentclass[prb,preprint,letterpaper,noeprint,longbibliography,nodoi,footinbib]{revtex4-1} 

% Note that AJP uses the same style as Phys. Rev. B (prb).
% The AIP Style Manual\cite{AIPstylemanual} is an indispensable reference on good physics writing, covering everything from planning and organization to standard spellings and abbreviations.
% Most important of all, please familiarize yourself with the AJP Statement of Editorial Policy,\cite{editorsite} which describes the types of manuscripts that AJP publishes and the audience for which AJP authors are expected to write.
% We look forward to receiving your submission to AJP.
\usepackage[utf8]{inputenc}
\usepackage{amsmath,amssymb,amsthm}
\usepackage{amsfonts}
\usepackage{graphicx}
\usepackage{float}
\usepackage{mathtools}
\usepackage{hyperref}

\bibliographystyle{apsrev4-2}
% \setlength{\parindent}{0pt}

\begin{document}

\title{GravExplain}

\author{James W. Gardner}
\email{u6069809@anu.edu.au}
\affiliation{Research School of Physics, Australian National University, 60 Mills Rd, Acton ACT 2601 Australia}

\author{Hannah Middleton}
\email{hannah.middleton@unimelb.edu.au}
\author{Andrew Melatos}
\email{amelatos@unimelb.edu.au}
\affiliation{School of Physics, University of Melbourne, Parkville, VIC, 3010, Australia}
\affiliation{OzGrav-Melbourne, Australian Research Council Centre of Excellence for Gravitational Wave Discovery}

% Summer 2019/202
\date{\today}

% AJP requires an abstract for all regular article submissions.
\begin{abstract}
GravExplain

\end{abstract}

\maketitle

\section{Introduction}
GravExplain


\section{Results}
GravExplain


\section{Conclusion}
GravExplain


\appendix
\section{}
GravExplain


\begin{acknowledgments}
We gratefully acknowledge Jude Prezens, Alex Tolotchkoc, and Blake Molyneux for their technical advice and generous assistance.

J.~Gardner also acknowledges the ANU PhB program for providing travel funding during the project.

This research is supported by the Australian Research Council Centre of Excellence for Gravitational Wave Discovery (OzGrav) (project number CE170100004) and the Institute of Physics International Member Grant. 


\end{acknowledgments}


\end{document}
