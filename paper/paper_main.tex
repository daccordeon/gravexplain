% GravExplain
%\documentclass[prb,reprint,nofootinbib]{revtex4-1} 
\documentclass[aps,pra,superscriptaddress,reprint]{revtex4-1}
% \documentclass[prb,preprint,letterpaper,noeprint,longbibliography,nodoi,footinbib]{revtex4-1} 

% Note that AJP uses the same style as Phys. Rev. B (prb).
% The AIP Style Manual\cite{AIPstylemanual} is an indispensable reference on good physics writing, covering everything from planning and organization to standard spellings and abbreviations.
% Most important of all, please familiarize yourself with the AJP Statement of Editorial Policy,\cite{editorsite} which describes the types of manuscripts that AJP publishes and the audience for which AJP authors are expected to write.
% We look forward to receiving your submission to AJP.
\usepackage[utf8]{inputenc}
\usepackage{amsmath,amssymb,amsthm}
\usepackage{amsfonts}
\usepackage{graphicx}
\usepackage{float}
\usepackage{mathtools}
\usepackage[usenames,dvipsnames]{xcolor}
\usepackage{hyperref}
% \usepackage{siunitx}
\usepackage{textcomp}
\usepackage{subfiles}
\usepackage{comment}
\usepackage{algpseudocode}
% \usepackage{algorithm}
% \usepackage{algorithmic}
\usepackage{lmodern}

\usepackage{silence}
\WarningFilter{revtex4-1}{Repair the float}

%for line number
%\usepackage{lineno}
%\linenumbers


% inner product
\newcommand{\ip}[2]{\langle #1,#2 \rangle}

\newcommand{\jam}{\textcolor{magenta}}
\newcommand{\han}{\textcolor{orange}}
\newcommand{\fixme}{\textcolor{blue}}


\begin{document}


\title{Continuous gravitational waves in the lab: recovering audio signals with a table-top optical microphone}

%\author{\textbf{[Authors removed for double-anonymous review.]}}



\author{James W. Gardner}
\email{u6069809@anu.edu.au}
% \affiliation{%
% College of Science, Australian National University, Acton, ACT, 2601, Australia 
% }%
\affiliation{%
ANU Centre for Gravitational Astrophysics, The Australian National University, Acton, ACT, 2601, Australia
}
\affiliation{%
OzGrav-ANU, Australian Research Council Centre of Excellence for Gravitational Wave Discovery, The Australian National University, Acton, ACT, 2601, Australia
} 
\affiliation{%
School of Physics, University of Melbourne, Parkville, Victoria, 3010, Australia
}


\author{Hannah Middleton}
\email{hannah.middleton@unimelb.edu.au}
\affiliation{%
School of Physics, University of Melbourne, Parkville, Victoria, 3010, Australia
}
\affiliation{%
Centre for Astrophysics and Supercomputing, Swinburne University of Technology, Hawthorn, Victoria 3122, Australia 
}
\affiliation{%
OzGrav-Melbourne, Australian Research Council Centre of Excellence for Gravitational Wave Discovery, Parkville, Victoria 3010, Australia
}

\author{Changrong Liu}
\affiliation{%
Department of Electrical and Electronic Engineering, University of Melbourne, Parkville, Victoria 3010, Australia
}

\author{Andrew Melatos}
\email{amelatos@unimelb.edu.au}
\affiliation{%
School of Physics, University of Melbourne, Parkville, Victoria, 3010, Australia
}
\affiliation{%
OzGrav-Melbourne, Australian Research Council Centre of Excellence for Gravitational Wave Discovery, Parkville, Victoria 3010, Australia
}


\author{Robin Evans}
\affiliation{%
Department of Electrical and Electronic Engineering, University of Melbourne, Parkville, Victoria 3010, Australia
}
\affiliation{%
OzGrav-Melbourne, Australian Research Council Centre of Excellence for Gravitational Wave Discovery, Parkville, Victoria 3010, Australia
}

\author{William Moran}
\affiliation{%
Department of Electrical and Electronic Engineering, University of Melbourne, Parkville, Victoria 3010, Australia
}


\author{Deeksha Beniwal}
\affiliation{%
Department of Physics, The University of Adelaide, South Australia, 5005, Australia
}
\affiliation{%
The Institute of Photonics and Advanced Sensing (IPAS), The University of Adelaide, South Australia, 5005, Australia
}
\affiliation{%
OzGrav-Adelaide, Australian Research Council Centre of Excellence for Gravitational Wave Discovery, South Australia, 5005, Australia
}


\author{Huy Tuong Cao}
\affiliation{%
Department of Physics, The University of Adelaide, South Australia, 5005, Australia
}
\affiliation{%
The Institute of Photonics and Advanced Sensing (IPAS), The University of Adelaide, South Australia, 5005, Australia
}
\affiliation{%
OzGrav-Adelaide, Australian Research Council Centre of Excellence for Gravitational Wave Discovery, South Australia, 5005, Australia
}

\author{Craig Ingram}
\affiliation{%
Department of Physics, The University of Adelaide, South Australia, 5005, Australia
}
\affiliation{%
The Institute of Photonics and Advanced Sensing (IPAS), The University of Adelaide, South Australia, 5005, Australia
}
\affiliation{%
OzGrav-Adelaide, Australian Research Council Centre of Excellence for Gravitational Wave Discovery, South Australia, 5005, Australia
}

\author{Daniel Brown}
\affiliation{%
Department of Physics, The University of Adelaide, South Australia, 5005, Australia
}
\affiliation{%
The Institute of Photonics and Advanced Sensing (IPAS), The University of Adelaide, South Australia, 5005, Australia
}
\affiliation{%
OzGrav-Adelaide, Australian Research Council Centre of Excellence for Gravitational Wave Discovery, South Australia, 5005, Australia
}

\author{Sebastian Ng}
\affiliation{%
Department of Physics, The University of Adelaide, South Australia, 5005, Australia
}
\affiliation{%
The Institute of Photonics and Advanced Sensing (IPAS), The University of Adelaide, South Australia, 5005, Australia
}
\affiliation{%
OzGrav-Adelaide, Australian Research Council Centre of Excellence for Gravitational Wave Discovery, South Australia, 5005, Australia
}






% Summer 2019/2020
\date{\today}


\begin{abstract}
Gravitational-wave observatories around the world are searching for continuous waves: persistent signals from sources such as spinning neutron stars. 
These searches use sophisticated statistical techniques to look for weak signals in noisy data. 
In this paper, we demonstrate these techniques using a table-top model gravitational-wave detector: a Michelson interferometer where sound is used as an analog for gravitational waves. 
Using signal processing techniques from continuous-wave searches, we demonstrate the recovery of tones with constant and wandering frequencies. 
We also explore the use of the interferometer as a teaching tool for educators in physics and electrical engineering by using it as an ``optical microphone'' to capture music and speech. 
% \jam{We mention undergraduate labs twice, roll this sentence into the end? Also, electrical sounds weird, why the change from electrical engineering?}
%The interferometer is used as an ``optical microphone'' to capture music and speech. 
% \jam{I changed all the \emph to ``'', feel free to change them back, but I think it is more explicit this way}
A range of filtering techniques used to recover signals from noisy data are detailed in the Supplementary Material. 
Here, we present highlights of our results using a combined notch plus Wiener filter and the statistical log minimum mean-square error (logMMSE) estimator. 
%We present a range of example filtering techniques to recover the original signal from noisy data, including a bandpass filter, notch filters, a Wiener filter, and the statistical log minimum mean-square-error (logMMSE) estimator.
% log minimum mean-square-error (logMMSE) estimator
Recordings of simple chords and drums are easily recovered using these techniques, but complex music and speech are more challenging. 
This demonstration can be used by educators in undergraduate laboratories and can be adapted for use as an engagement tool for communicating gravitational-wave and signal-processing topics to non-specialist audiences. 
% This demonstration has scope for use by educators in undergraduate laboratories and can be adapted for use as an engagement tool for communicating gravitational-wave and signal-processing topics to non-specialist audiences. 
% rearranged in response to Anna's pnp review
%Finally we use the interferometer as an `optical microphone' to capture music and speech and apply filtering techniques to recover the original signals from the noisy data. Such techniques include simple bandpass filtering, notch filters, a Weiner filter, and the statistical logMMSE method.  
%Simple chords and drums are easily recovered, but complex music and speech are more challenging. 
%This demonstration has scope for use in physics and electrical engineering undergraduate laboratories. 
%It can also be adapted for use as an engagement tool for communicating gravitational wave and signal-processing topics to non-specialist audiences. 
\end{abstract}



\maketitle

\section{Introduction}
\label{sec:introduction}
\subfile{ifo-introduction.tex}


\section{Table-top gravitational-wave science}
\label{sec:ifo}
\subfile{ifo-tableTopGWs.tex}


\section{Constant frequency signal}
\label{sec:single_tone}
\subfile{ifo-singleTone.tex}

 
\section{Wandering frequency signal}
\label{sec:viterbi_wandering}
\subfile{ifo-wanderingTone.tex}


\section{Complex audio: music and speech}
\label{sec:optical_microphone}
\subfile{ifo-complexAudio.tex}


\section{Future work}
\label{sec:future_work}
\subfile{ifo-futureWork.tex}


\section{Conclusions}
\label{sec:conclusions}
\subfile{ifo-conclusions.tex}





\begin{acknowledgments}
%\textbf{[Removed for double-anonymous review.]} 
The authors are grateful to Jude Prezens, Alex Tolotchkoc, and Blake Molyneux for their technical advice and generous assistance; Patrick Meyers, Margaret Millhouse, Sofia Suvorova for useful discussions; Patrick Clearwater, Patrick Meyers, Suk Yee Yong, Lucy Strang, Julian Carlin, Sanjaykumar Patil, and Alex Cameron for early work on the interferometer design requirements and construction; and the LIGO Education Public Outreach working group, in particular, Anna Green, Lynn Cominsky, and Sam Cooper, for their helpful feedback and suggestions.  
This research is supported by the Australian Research Council Centre of Excellence for Gravitational Wave Discovery (OzGrav) (project number CE170100004). 
Financial support towards hardware was provided by the Institute of Physics International Member Grant and the OzGrav Outreach Support scheme. 
Travel support was provided by the Australian National University PhB Science program.
This work has been assigned LIGO document number P2000386.

\end{acknowledgments}


\appendix
\subfile{ifo-appendix.tex}


% \nocite{*}
\bibliographystyle{myunsrt}
\bibliography{ifoDemoBib}




\end{document}

