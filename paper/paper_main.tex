% GravExplain
%\documentclass[prb,reprint,nofootinbib]{revtex4-1} 
\documentclass[pra,superscriptaddress,reprint,amsmath,amssymb,nofootinbib]{revtex4-1}
% \documentclass[prb,preprint,letterpaper,noeprint,longbibliography,nodoi,footinbib]{revtex4-1} 

% Note that AJP uses the same style as Phys. Rev. B (prb).
% The AIP Style Manual\cite{AIPstylemanual} is an indispensable reference on good physics writing, covering everything from planning and organization to standard spellings and abbreviations.
% Most important of all, please familiarize yourself with the AJP Statement of Editorial Policy,\cite{editorsite} which describes the types of manuscripts that AJP publishes and the audience for which AJP authors are expected to write.
% We look forward to receiving your submission to AJP.
\usepackage[utf8]{inputenc}
\usepackage{amsmath,amssymb,amsthm}
\usepackage{amsfonts}
\usepackage{graphicx}
\usepackage{float}
\usepackage{mathtools}
\usepackage[usenames,dvipsnames]{xcolor}
\usepackage{hyperref}
% \usepackage{siunitx}
\usepackage{textcomp}
\usepackage{subfiles}
\usepackage[bottom]{footmisc}
\usepackage{comment}
\usepackage{algpseudocode}
% \usepackage{algorithm}
% \usepackage{algorithmic}

\usepackage{lineno}
\linenumbers

%\usepackage{dcolumn}
%\usepackage{bm}
%\usepackage[pdftex,plainpages=false,colorlinks=true]{hyperref}
%\usepackage[toc,page,title,titletoc,header]{appendix} 
%\usepackage{multirow}
%\usepackage{tabularx}
%\usepackage[bottom]{footmisc}
%\usepackage{caption}
%\usepackage{subcaption}
%\usepackage[bottom]{footmisc}
%\usepackage{soul}

%\bibliographystyle{apsrev4-2}
% \setlength{\parindent}{0pt}

\newcommand{\jam}{\textcolor{magenta}}
\newcommand{\han}{\textcolor{orange}}




\begin{document}

%\title{GravExplain: Demonstrating continuous gravitational wave searches using sound and a table-top interferometer}
\title{Continuous gravitational waves in the lab: recovering audio signals with a table-top optical microphone}





\author{James W. Gardner}
\email{u6069809@anu.edu.au}
\affiliation{%
College of Science, Australian National University, Acton, ACT, 2601, Australia 
}%
\affiliation{%
 School of Physics, University of Melbourne, Parkville, Victoria, 3010, Australia
}


\author{Hannah Middleton}
\email{hannah.middleton@unimelb.edu.au}
\affiliation{%
 School of Physics, University of Melbourne, Parkville, Victoria, 3010, Australia
}
\affiliation{%
OzGrav-Melbourne, Australian Research Council Centre of Excellence for Gravitational Wave Discovery, Parkville, Victoria 3010, Australia
}

\author{Andrew Melatos}
\email{amelatos@unimelb.edu.au}
\affiliation{%
 School of Physics, University of Melbourne, Parkville, Victoria, 3010, Australia
}
\affiliation{%
OzGrav-Melbourne, Australian Research Council Centre of Excellence for Gravitational Wave Discovery, Parkville, Victoria 3010, Australia
}

\author{Changrong Liu}
\affiliation{%
Department of Electrical and Electronic Engineering, University of Melbourne, Parkville, Victoria 3010, Australia
}


\author{Robin Evans}
\affiliation{%
Department of Electrical and Electronic Engineering, University of Melbourne, Parkville, Victoria 3010, Australia
}
\affiliation{%
OzGrav-Melbourne, Australian Research Council Centre of Excellence for Gravitational Wave Discovery, Parkville, Victoria 3010, Australia
}

\author{William Moran}
\affiliation{%
Department of Electrical and Electronic Engineering, University of Melbourne, Parkville, Victoria 3010, Australia
}


\author{Deeksha Beniwal}
\affiliation{%
Department of Physics and The Institute of Photonics and Advanced Sensing (IPAS), The University of Adelaide, South Australia, 5005, Australia
}
\affiliation{%
OzGrav-Adelaide, Australian Research Council Centre of Excellence for Gravitational Wave Discovery, South Australia, 5005, Australia
}


\author{Huy Tuong Cao}
\affiliation{%
Department of Physics and The Institute of Photonics and Advanced Sensing (IPAS), The University of Adelaide, South Australia, 5005, Australia
}
\affiliation{%
OzGrav-Adelaide, Australian Research Council Centre of Excellence for Gravitational Wave Discovery, South Australia, 5005, Australia
}

\author{Craig Ingram}
\affiliation{%
Department of Physics and The Institute of Photonics and Advanced Sensing (IPAS), The University of Adelaide, South Australia, 5005, Australia
}
\affiliation{%
OzGrav-Adelaide, Australian Research Council Centre of Excellence for Gravitational Wave Discovery, South Australia, 5005, Australia
}

\author{Daniel Brown}
\affiliation{%
Department of Physics and The Institute of Photonics and Advanced Sensing (IPAS), The University of Adelaide, South Australia, 5005, Australia
}
\affiliation{%
OzGrav-Adelaide, Australian Research Council Centre of Excellence for Gravitational Wave Discovery, South Australia, 5005, Australia
}

\author{Sebastian Ng}
\affiliation{%
Department of Physics and The Institute of Photonics and Advanced Sensing (IPAS), The University of Adelaide, South Australia, 5005, Australia
}
\affiliation{%
OzGrav-Adelaide, Australian Research Council Centre of Excellence for Gravitational Wave Discovery, South Australia, 5005, Australia
}

\author{\han{to do: check names - initials?}}
\affiliation{%
\han{to do: check all affiliations}
}

% Summer 2019/2020
\date{\today}


\begin{abstract}
Gravitational-wave observatories around the world are searching for continuous waves: persistent signals from spinning neutron stars. 
These searches use sophisticated statistical techniques to look for weak signals in noisy data. 
In this paper, we demonstrate these techniques using a table-top model gravitational-wave detector (a Michelson interferometer), where sound plays the role of gravitational waves. 
Using signal processing techniques from continuous-wave searches we demonstrate the recovery of tones with constant and wandering frequencies. 
Finally we use the interferometer as an `optical microphone' to capture music and speech and apply filtering techniques to recover the original signals from the noisy data. Such techniques include simple bandpass and notch filters, as well as comparing a Weiner filter to the statistical logMMSE method.  
Simple chords and drums are easily recovered, but complex music and speech are more challenging. 
This demonstration has scope for use in physics and electrical engineering undergraduate laboratories. 
It can also be adapted for use as an engagement tool for communicating gravitational-wave and signal-processing topics to non-specialist audiences. 
\end{abstract}



\maketitle

\section{Introduction}
\label{sec:introduction}
\subfile{ifo-introduction.tex}


\section{Table-top gravitational wave science}
\label{sec:ifo}
\subfile{ifo-tableTopGWs.tex}


\section{Constant tone}
\label{sec:single_tone}
\subfile{ifo-singleTone.tex}

 
\section{Wandering tone}
\label{sec:viterbi_wandering}
\subfile{ifo-wanderingTone.tex}


\section{Complex audio: music and speech}
\label{sec:optical_microphone}
\subfile{ifo-complexAudio.tex}


\section{Future work}
\label{sec:future_work}
\subfile{ifo-futureWork.tex}


\section{Conclusions}
\label{sec:conclusions}
\subfile{ifo-conclusions.tex}





\begin{acknowledgments}
The authors are grateful to Jude Prezens, Alex Tolotchkoc, and Blake Molyneux for their technical advice and generous assistance throughout the project. 
This research is supported by the Australian Research Council Centre of Excellence for Gravitational Wave Discovery (OzGrav) (project number CE170100004). 
This work also received financial support for hardware from the Institute of Physics International Member Grant and for travel from the Australian National University PhB Science program.

\end{acknowledgments}


\appendix
\subfile{ifo-appendix.tex}


\nocite{*}
\bibliographystyle{myunsrt}
\bibliography{ifoDemoBib}




\end{document}

