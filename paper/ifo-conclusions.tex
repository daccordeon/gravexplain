\documentclass[paper-main.tex]{subfiles}


\begin{document}




In this paper, we use a table-top Michelson interferometer as an analogue to a gravitational-wave detector, demonstrating a range of signal processing techniques used within the gravitational-wave community and more broadly. Gravitational-wave signals are simulated by playing sound recordings through a speaker fixed to the back of one of the arm mirrors. We consider quasi-monochromatic signals as an analogue to continuous wave signals as well as short segments of speech and music for more general interest. These recordings of complex audio could also serve as more distant analogues to un-modelled gravitational-wave bursts, e.g. from supernovae.  

% reduced detail for conclusion (trying to reduce overall length)
In Sections~\ref{sec:single_tone} and~\ref{sec:viterbi_wandering} we analyse audio signals with constant and wandering frequencies respectively as analogues to continuous gravitational wave signals. 
The wandering frequency signal is recovered using using the Viterbi algorithm. 
These sections directly demonstrate the techniques used in current continuous gravitational-wave searches. 
%We demonstrate the recovery of a signal with constant frequency and signal with slowly wandering frequency in Sections~\ref{sec:ifo} and ~\ref{sec:single_tone}, respectively. We use a basic webcam with a sampling rate of $30\,{\rm Hz}$ to track the interference pattern. For the wandering frequency we successfully recover the injected signal using the Viterbi algorithm. These sections directly demonstrate the techniques used in continuous gravitational-wave searches.  

% again reducing overall length as we probably don't need to list everything
The recovery of more complex music and speech signals are demonstrated in Section~\ref{sec:optical_microphone}, where we treat the interferometer as an optical microphone. 
Simple music recordings can be recognised when played back, however speech remains unintelligible after applying a selection of filters and statistical techniques from signal processing. 
We find that the logMMSE estimator produces the best result of those applied here, however it does not enhance the recording to an intelligible level. 
This demonstration allows students to explore a variety of filtering and speech enhancement techniques. 
Further avenues of exploration to improve intelligibility are detailed in Section~\ref{sec:future_work}. 
%In Section~\ref{sec:optical_microphone}, we demonstrate the recovery of more complex signals of music and speech. By substituting the webcam with a photodiode read by a Raspberry Pi, we transform the interferometer into an optical microphone by greatly increasing the sampling rate. We demonstrate that the optical microphone is capable of playing back simple recordings well but produces unintelligible recordings of speech. The mains signal at $50\,{\rm Hz}$ and its harmonics dominate the photodiode output. To recover the signal from under this noise we apply a hierarchy of filters and statistical techniques from signals processing, including a Butterworth band-pass filter and a notch filter, to a Wiener filter and the logMMSE estimator. We find that the logMMSE estimator works the best from those tried but still doesn’t enhance the speech to an intelligible level. This demonstration allows students to explore a variety of filtering and speech enhancement techniques. Further avenues of exploration include even more advanced processing techniques and a full characterisation of the transfer function of the system, with the goal to lead to improvements in intelligibility. 

The demonstration presented has potential to be adapted for use in the undergraduate laboratory.
It can be used to teach topics of interest to physics and electrical engineering students including, interferometer physics; gravitational-wave detection, searches, and analysis; and signal processing with filters and speech enhancement techniques. 
The increased excitement and public interest in the field of gravitational-wave research in recent years means that this demonstration may also be adapted as a tool for explaining gravitational-wave research to a wider, non-specialist audience. 
The field of gravitational-wave astrophysics continuous to grow and the future will bring many more detections of binary black holes and neutron stars, as well as the anticipated first detection of other classes of signals such as continuous waves to which this demonstration provides some charming insights. 
%The demonstration presented here has great potential to be adapted for use in the undergraduate laboratory. It can be used to demonstrate topics such as interferometer physics; gravitational-wave detection, searches, and analysis; and signal processing with filters and speech enhancement techniques to physics and electrical engineering students. Given the increased level of excitement and public interest in the field of gravitational-wave astronomy in recent years, this table-top demonstration can be used as a tool for explaining gravitational-wave research to a wider, non-specialist audience. The future will bring many more detections, including further observations of transient signals from binary black hole and binary neutron star mergers, as well as the anticipated first detections of other classes of signal, including continuous waves, to which this demonstration provides some charming insights.


% Continuous gravitational wave searches use sophisticated statistical techniques to search for weak signals in noisy interferometric data. We demonstrate some of these techniques by observing sound with a table-top Michelson interferometer. In particular, we successfully use the Viterbi algorithm to recover a tone with wandering frequency from a webcam video of the interference pattern.

%Using a webcam to track the interference pattern, we inject a wandering frequency signal into the system with a speaker attached to the back of one of mirrors of the interferometer. We then demonstrate how to recover this signal using the Viterbi algorithm, just as is done in continuous gravitational wave searches.

%These demonstrations have good potential to be adapted into the undergraduate laboratory. There is much to explore in the physics of the interferometer and the astronomy of the motivating continuous gravitational wave searches. Most of all, the signal processing of speech enhancement is an interesting problem that’s relevant to digital communications.


\end{document}
