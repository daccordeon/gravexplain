\documentclass[paper-main.tex]{subfiles}


\begin{document}

% reduced conclusion 


In this paper, we use a table-top Michelson interferometer as an analog to a gravitational-wave detector, demonstrating signal processing techniques used within the gravitational-wave community.
We explore the use of the interferometer as an optical microphone and consider a more general treatment of signal processing with complex audio signals, which can also serve as a distant analog to minimally modelled gravitational-wave burst signals, e.g., from supernovae.
% justify ``un-modeled''? - james
% fair point, it's the lingo used in LIGO, but really it's more accurate to say minimally modelled - Han
The demonstration can be adapted for use in both the physics and engineering undergraduate laboratory, providing opportunities for cross-disciplinary teaching. 
Additionally, it can be used as a tool for explaining gravitational-wave research to a wider, non-specialist audience. 




As the field of gravitational-wave astrophysics continues to grow, the future will bring many more detections of binary black holes and neutron stars, as well as the anticipated first detection of other classes of signals, such as continuous waves, to which this demonstration provides some charming insights. 

\end{document}
