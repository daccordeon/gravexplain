\documentclass[paper-main.tex]{subfiles}


\begin{document}

% reduced conclusion 


In this paper, we use a table-top Michelson interferometer as an analog to a gravitational-wave detector, demonstrating signal processing techniques used within the gravitational-wave community.
We explore the use of the interferometer as an optical microphone and consider a more general treatment of signal processing with complex audio signals, which can also serve as a distant analog to un-modeled gravitational-wave burst signals, e.g., from supernovae.
% justify ``un-modeled''? - james
The demonstration can be adapted for use in the undergraduate laboratory and as a tool for explaining gravitational-wave research to a wider, non-specialist audience. 


\begin{comment}

The demonstration presented has the potential to be adapted for use in the undergraduate laboratory.
It can be used to teach topics of interest to physics and electrical engineering students including interferometer physics; gravitational-wave detection, searches, and analysis; and signal processing with filters and speech enhancement techniques. 
The increased excitement and public interest in the field of gravitational-wave research in recent years mean that this demonstration may also be adapted as a tool for explaining gravitational-wave research to a wider, non-specialist audience. 

\end{comment}

As the field of gravitational-wave astrophysics continues to grow, the future will bring many more detections of binary black holes and neutron stars, as well as the anticipated first detection of other classes of signals, such as continuous waves, to which this demonstration provides some charming insights. 

\end{document}
