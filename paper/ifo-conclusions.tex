\documentclass[paper-main.tex]{subfiles}


\begin{document}


% Continuous gravitational wave searches use sophisticated statistical techniques to search for weak signals in noisy interferometric data. We demonstrate some of these techniques by observing sound with a table-top Michelson interferometer. In particular, we successfully use the Viterbi algorithm to recover a tone with wandering frequency from a webcam video of the interference pattern.

In this paper we present an educational demonstration in which an \han{inexpensive} \jam{[see comments in introduction]}, table top Michelson interferometer serves as an analogue to a gravitational-wave detector. 
Vibrations produced by gravitational-wave signals are simulated with sound recordings played though a speaker fixed to the back of one of the interferometer mirrors. 
We consider quasi-monochromatic signals as an analogue to continuous wave signals as well as short segments of speech and music which are (more distant) analogues of unmodelled gravitational-wave bursts, e.g. from supernovae. 
% This serves to demonstrate signal processing techniques.


In Sections~\ref{sec:ifo} and ~\ref{sec:single_tone} we demonstrate the recovery of signals with constant and slowly wandering frequency. 
A webcam sampling at $30\,{\rm Hz}$ is used to track the interference pattern and the injected signal is recovered succesfully with the Viterbi algorithm. 
These sections direcly demonstrate the continuous wave search techniques used in gravitational-wave data analysis. 
%Using a webcam to track the interference pattern, we inject a wandering frequency signal into the system with a speaker attached to the back of one of mirrors of the interferometer. We then demonstrate how to recover this signal using the Viterbi algorithm, just as is done in continuous gravitational wave searches.


In Section~\ref{sec:optical_microphone} we demonstrate recovery of more complex signals of music and speech. 
Substituting the webcam for the increased frequency response of a photodiode read by a Raspberry Pi, we transform the interferometer into an optical microphone.
\jam{[Confirming Hannah's comment, the Pi was only necessary to read the photodiode, the webcam video I just saved onto PC and analysed directly]}
We demonstrate that the optical microphone is capable of playing back simple recordings well but that it produces unintelligible recordings of speech. 
Mains hum at $50\,{\rm Hz}$ and its harmonics dominates the photodiode output. 
This offers students the opportunity to explore speech enhancement techniques to remove it, including the well performing logMMSE estimator. 
Further avenues of exploration include a more complete model of the optical microphone transfer function, which may lead to improvements in intelligibility. 


The demonstration presented here has good potential to be adapted for use in the undergraduate laboratory. 
It can be used to demonstrate topics of interest to physicists and electrical engineers including: interferometer physics; gravitational-wave detection, searches, and analysis; and signal processing with filters and speech enhancement. 
The detection of gravitational waves in 2015 led to increased excitement and public interest in gravitational wave science. 
The table-top demonstration presented here can also be adapted as a tool to explain gravitational wave research developments for a wider non-specialist audience. 
The future will bring many more detections, including further observations of transient signals from binary black hole and binary neutron star mergers, as well as the anticipated first detections of other classes of signal, including continuous waves. 

%These demonstrations have good potential to be adapted into the undergraduate laboratory. There is much to explore in the physics of the interferometer and the astronomy of the motivating continuous gravitational wave searches. Most of all, the signal processing of speech enhancement is an interesting problem that’s relevant to digital communications.


\end{document}
