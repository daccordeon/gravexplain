\documentclass[paper-main.tex]{subfiles}

\begin{document}


% general gw intro
In 2015, gravitational waves were observed for the first time from the merger of two black holes in a binary system~\cite{GW150914}. 
The observation, made by the Laser Interferometer Gravitational-wave Observatory~\citep[LIGO]{AdvancedLIGO:2015}, marked a breakthrough in modern astrophysics and revealed a new means to observe the universe. 
Since 2015, the LIGO and Virgo~\cite{AdvancedVirgo:2015} observatories have made numerous detections of binary black hole~\cite{GW151226,GW170104,GW170814} and binary neutron star~\cite{GW170817,GW170817multi,GW190425} mergers. 
In recent years, there has been increased public interest in gravitational wave science. 
Many gravitational wave research groups around the world have produced demonstrations and activities to explain the topic to a general audience.
Activities range from hands-on demonstrations, exhibitions~\cite{TTExhibit:2020,LIGOScienceEducationCentre:online,GravityDiscoveryCentre:online}, and online data analysis tutorials~\cite{GWOSC:online,LOSC:2015}, to phone apps~\cite{LaserLabs:online,SciVR:online,chirp:online}, online games~\cite{BlackHoleHunter:online}, and musical performances~\cite{ArthurJeffesMusic:online,GravitySynthLeonTrimble:online}%\han{look up Sarah Farmer's project}.

% what are gws and how do we detect them...
Gravitational waves are a prediction of Einstein's theory of general relativity. 
They are disturbances in space-time caused by the acceleration of massive objects. 
The effect of gravitational waves is a change in lengths; a `stretching and squeezing' of the distance between objects.
Observatories such as LIGO, Virgo, and KAGRA~\cite{KAGRA:2013} are laser interferometers; they use the interference of laser light to measure changes in distance. 
%As shown in an aerial view of LIGO Hanford in Figure~\ref{fig:ligo_pic}, they use beam arms kilometres long.
These observatories are extremely complex, but are fundamentally are based on the Michelson interferometer. 
Table-top Michelson interferometers are commonly used in undergraduate lab experiments~\cite{UgoliniEtAl:2019} and to demonstrate the science of gravitational wave detection to non-specialist audiences~\cite{ThorLabsIFO,NikhefIFO,TTExhibit:2020,LIGOIFOGlue,LIGOIFOMagnets}.

% continuous gws
To date the network of gravitational wave observatories have observed short-duration transient signals~\cite{GWTC-1:2018,GWOSC:online}. 
However, the network is also searching for continuous gravitational waves; persistent periodic signals at near-monochromatic frequencies.
Continuous waves may be emitted at constant frequencies, or may wander slowly in frequency over time. 
One of the prime targets of continuous wave searches are rotating neutron stars in low mass X-ray binaries (LMXBs).
LMXBs are binary systems containing a compact object (such as a neutron star or a black hole) in orbit with a low mass stellar companion~\cite{xraybinaries:1997}. 
The LMXB Scorpius X-1 is an LMXB which is particularly bright in X-rays. 
It is a prime target for continuous wave searches from LMXBs and numerous, as yet unsuccessful, searches have been performed~\cite{ScoX1O2Viterbi:2019,RadiometerO1O2:2019,SearchCrossCorrO1:2017}.

% sound analogy
Here, we demonstrate the use of a Michelson interferometer as a tool to explain the search for continuous waves to a general audience.
The analogy of sound is commonly used in explaining gravitational wave observations. 
%; the analogy of listening to the universe is often used to explain gravitational wave observations.
%; transient gravitational wave signals have been converted into audio.
In this work we utilise this analogy by playing sound close to a Michelson interferometer, simulating a gravitational-wave signal. 
Audio of constant or slowly wandering frequency tones provide an analogue of a continuous wave signal.
Continuous wave data analysis techniques are also demonstrated. 
The techniques used in this work are similar to those used to search for continuous waves in LIGO and Virgo data~\cite{ScoX1O2Viterbi:2019,ScoX1ViterbiO1:2017,SuvorovaEtAl:2017,SuvorovaEtAl:2017}, particularly for the wandering frequency demonstration. 


This demonstration can be used to explain the search for continuous waves to a general audience as well as undergraduate students. 
The analysis methods applied here follow signal processing techniques used in some of the most recent searches for continuous waves~\cite{ScoX1O2Viterbi:2019}. 
As an undergraduate lab experiment, this demonstration covers topics of interest to both physics and engineering students. 


In this work we detail our table-top interferometer design in Section~\ref{sec:ifo} and describe observing a single note from a speaker in Section~\ref{sec:single_tone}. In Section~\ref{sec:viterbi_wandering}, we demonstrate the analysis techniques for observing a wandering frequency signal.
In Section~\ref{sec:optical_microphone}, we demonstrate capturing and playing back complex audio, such as music and speech. This demonstration of an `optical microphone' serves as a more general exhibition of signal processing. We suggest avenues of future work in Section~\ref{sec:future_work} and draw conclusions in Section~\ref{sec:conclusions}.



\end{document}
