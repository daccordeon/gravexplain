\documentclass[paper-main.tex]{subfiles}

\begin{document}


% general gw intro
In 2015, gravitational waves were observed for the first time from the merger of two black holes~\cite{GW150914}. 
The observation, made by the Laser Interferometer Gravitational-wave Observatory~\citep[LIGO]{AdvancedLIGO:2015}, marked a breakthrough in modern astrophysics and revealed a new means to observe the universe. 
Since 2015, the LIGO and Virgo~\cite{AdvancedVirgo:2015} observatories have made numerous detections of binary black hole~\cite{GW151226,GW170104,GW170814} and binary neutron star~\cite{GW170817,GW170817multi,GW190425} mergers. 
Gravitational-wave detection has led to increased public and media interest in gravitational-wave science. 
Gravitational-wave research groups around the world have produced demonstrations and activities to explain their research to general audiences. 
Activities range from hands-on demonstrations, exhibitions~\cite{TTExhibit:2020,LIGOScienceEducationCentre:online,GravityDiscoveryCentre:online}, and online data analysis tutorials~\cite{GWOSC:online,LOSC:2015}, to phone apps~\cite{LaserLabs:online,SciVR:online,chirp:online}, online games~\cite{BlackHoleHunter:online}, and musical performances~\cite{ArthurJeffesMusic:online,GravitySynthLeonTrimble:online}%\han{look up Sarah Farmer's project}.

% what are gws and how do we detect them...
Gravitational waves are a prediction of Albert Einstein's theory of general relativity~\cite{Einstein:1916}. 
They are disturbances in spacetime caused by the acceleration of asymmetric massive objects. 
The effect of gravitational waves is a change in lengths: a `stretching and squeezing' of the distance between objects.
Observatories such as LIGO, Virgo, and KAGRA~\cite{KAGRA:2013} are laser interferometers.
They use the interference of laser light to measure changes in distance. 
%As shown in an aerial view of LIGO Hanford in Figure~\ref{fig:ligo_pic}, they use beam arms kilometres long.
Gravitational-wave observatories are extremely complex but are fundamentally based on the Michelson interferometer. 
Table-top Michelson interferometers are commonly used in undergraduate lab experiments~\cite{UgoliniEtAl:2019} and to demonstrate the science of gravitational wave detection to non-specialist audiences~\cite{ThorLabsIFO,NikhefIFO,TTExhibit:2020,LIGOIFOGlue,LIGOIFOMagnets}.

% continuous gws
To date the network of gravitational wave observatories has observed short-duration transient signals~\cite{GWTC-1:2018,GWOSC:online}. 
However, the network is also searching for continuous gravitational waves: persistent, periodic, near-monochromatic signals.
Continuous waves may be emitted at constant frequencies, or may wander slowly in frequency over time. 
For example, rotating neutron stars in low mass X-ray binaries (LMXBs) are one of the targets for continuous-wave searches. 
LMXBs are binary systems containing a compact object (such as a neutron star or a black hole) in orbit with a low mass stellar companion, where the rotation frequency of the compact object wanders over time due to variable accretion of matter (and hence angular momentum transfer) from the companion~\cite{xraybinaries:1997}. 
Scorpius X-1 is an LMXB which is particularly bright in X-rays. 
It is a prime target for continuous wave searches from LMXBs and numerous, as yet unsuccessful, searches have been performed~\cite{ScoX1O2Viterbi:2019,RadiometerO1O2:2019,SearchCrossCorrO1:2017}.


% sound analogy
In this paper, we present a table-top experiment, where a standard Michelson interferometer plays the role of a `toy' gravitational-wave detector, which detects acoustic vibrations (such as music) instead of gravitational waves. 
As an undergraduate lab experiment this demonstration can be used to teach topics ranging from gravitational-wave detection and analysis to electronics, signal processing, and speech enhancement. 
It gives students in courses such as physics and electrical engineering the freedom to explore the response of an accessible yet nontrivial opto-mechanical system using a hierarchy of data analysis techniques of increasing complexity, including those used in the search for continuous waves in LIGO and Virgo data~\cite{ScoX1O2Viterbi:2019,ScoX1ViterbiO1:2017,SuvorovaEtAl:2017,SuvorovaEtAl:2017}.
The demonstration can also be adapted for use as an outreach tool to explain gravitational-wave science to a general audience. 


The paper presents 
In this work we present a selection of demonstrations using different types of audio signals ranging from simple tones, analogous to continuous wave signals, to more complex audio of music and speech. 
In Section~\ref{sec:ifo} we detail the table-top interferometer design. 
We demonstrate observing a single note from a speaker in Section~\ref{sec:single_tone}.
In Section~\ref{sec:viterbi_wandering} we replace the single note with a slowly changing frequency and analyse the signal using a hidden Markov model technique from continuous-wave searches. 
In Section~\ref{sec:optical_microphone}, we demonstrate capturing and playing back complex audio, such as music and speech. 
This demonstration of an `optical microphone' serves as a more general exhibition of signal processing. 
We suggest avenues of future work in Section~\ref{sec:future_work} and draw conclusions in Section~\ref{sec:conclusions}.
Appendix~\ref{app:code} collects the software and scripts used to produce this work. 



\end{document}
