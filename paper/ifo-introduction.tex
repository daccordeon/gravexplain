\documentclass[paper-main.tex]{subfiles}

\begin{document}

In 2015, the first detection of gravitational waves from the merger of two black holes marked a breakthrough in modern astrophysics and revealed a new means to observe the Universe~\cite{GW150914}.
Gravitational waves are a prediction of Albert Einstein's theory of General Relativity~\cite{Einstein:1916}, 
They are disturbances in spacetime caused by the acceleration of asymmetric massive objects.
The effect of gravitational waves is a change in lengths: a ``stretching and squashing'' of the distance between objects. 
Ground-based gravitational-wave observatories such as the Advanced Laser Interferometer Gravitational-wave Observatory (LIGO~\cite{AdvancedLIGO:2015}), Advanced Virgo~\cite{AdvancedVirgo:2015}, GEO600~\cite{GEO600:2010}, and KAGRA~\cite{KAGRA:2013} use the interference of laser light to measure changes in distance. 
These observatories are extremely complex but are fundamentally based on the Michelson interferometer. 
Table-top interferometers are commonly used in undergraduate laboratory experiments~\cite{UgoliniEtAl:2019} and to demonstrate the science of gravitational-wave detection to non-specialist audiences~\cite{ThorLabsIFO,NikhefIFO,AMIGO:online,TTExhibit:2020,LIGOIFOGlue,LIGOIFOMagnets}.


% CW 
To date, the network of gravitational-wave observatories has observed short-duration transient signals from the mergers of binary black holes~\cite{GW150914,GW151226,GW170104,GW170814,GW190521,GWTC-1:2018,GWTC-2:2020} and binary neutron stars~\cite{GW170817,GW170817multi,GW190425,GWTC-1:2018,GWTC-2:2020}, as well as a black hole merger with an object that could either have been a neutron star or another black hole~\cite{GW190814}. 
However, the network is also searching for continuous gravitational waves: persistent, periodic, near-monochromatic signals, which are yet to be detected. 
Continuous waves may be emitted at a constant frequency or may wander slowly in frequency over time. 
Rotating neutron stars are prime candidates for continuous-wave emission, especially those in low mass X-ray binaries (LMXB), where the neutron star is in orbit with a low mass stellar companion.
%where a compact object such as a black hole or a neutron star is in orbit with a low mass stellar companion.
% we're talking about neutron stars in LMXB's, so we don't need to mention black holes in LMXB's.
The rotation frequency of the neutron star in an LMXB can wander over time due to variable accretion of matter (and hence angular momentum transfer) from the stellar companion~\cite{xraybinaries:1997}. 
Scorpius X-1 is a prime LMXB target for continuous-wave searches. 
%Scorpius X-1 is an LMXB that is particularly bright in X-rays, making it a prime target for continuous wave searches. %from LMXBs. 
Numerous, as yet unsuccessful, searches have been performed for Scorpius X-1 and other LMXBs~\cite{ScoX1O2Viterbi:2019, ScoX1ViterbiO1:2017, RadiometerO1O2:2019, SearchRadiometerO1:2017, SearchCrossCorrO1:2017, ScoX1ViterbiO1:2017, SearchTwoSpecS6:2017, MiddletonEtAlO2LMXBs:2020}. 



% changed present to use, as this is not the first time a Michelson has been presented as a tpy GW detector - hannah
In this paper, we use a table-top Michelson interferometer as a toy gravitational-wave detector designed to detect sound instead of gravitational waves. 
We then extend its use to an ``optical microphone'', using light to capture sound, and present a range of example analysis techniques for educators to use. 
% re-added "as an undergraduate experiment to be clear we are not talking about school teaching - hannah
As an undergraduate lab experiment, the demonstration can be used to teach topics ranging from continuous-wave detection and analysis to electronics, signal processing, and speech enhancement. 
%% I don't like the use of ``freedom to explore''. 
% removed Additionally - I don't think it is neccessary - hannah
It allows students in courses such as physics and electrical engineering to explore the response of an accessible, yet nontrivial, optomechanical system using a hierarchy of data analysis techniques of increasing complexity, including those used in the search for continuous waves in LIGO and Virgo data~
\cite{SuvorovaEtAl:2017,SuvorovaEtAl:2016, ScoX1O2Viterbi:2019, ScoX1ViterbiO1:2017, BeniwalEtAl:2021, MillhouseStrangMelatos:2020, JonesSun:2020, MiddletonEtAlO2LMXBs:2020, PostMergerRemnantSearch:2019, BayleyEtAlSOAP:2019, SunEtAlSNR:2018}. 
% This demonstration can be adapted to an effective outreach tool to explain gravitational wave science to a non-specialist audience.
% Rephrased as this kind of sounds like we are an exhibition too [hannah] - This demonstration also has the potential to be used as an outreach tool alongside other exhibitions
%This demonstration also has the potential to be used as an outreach tool alongside a range of other public engagement demonstrations and activities including; hands-on demonstrations and exhibitions~\cite{TTExhibit:2020,CavagliaExhibit:2009,LIGOScienceEducationCentre:online,GravityDiscoveryCentre:online}, online data analysis tutorials~\cite{GWOSC:online,LOSC:2015}, phone applications~\cite{LaserLabs:online,SciVR:online,chirp:online}, online games~\cite{BlackHoleHunter:online}, and musical performances~\cite{ArthurJeffesMusic:online,GravitySynthLeonTrimble:online} which have been developed by gravitational-wave research groups around the world. 
This demonstration also has the potential to be used as an outreach tool alongside a range of other public engagement demonstrations and activities developed by gravitational-wave research groups around the world~\cite{TTExhibit:2020,CavagliaExhibit:2009, LIGOScienceEducationCentre:online, GravityDiscoveryCentre:online, GWOSC:online, LOSC:2015, LaserLabs:online, SciVR:online,chirp:online, BlackHoleHunter:online, ArthurJeffesMusic:online, GravitySynthLeonTrimble:online}. 
These tools allow scientists to cater to the increased public and media interest in this field and explain gravitational-wave science to non-specialist audiences.
%\han{look up Sarah Farmer's project}


This paper is laid out as follows. 
In Section~\ref{sec:ifo}, we detail the table-top interferometer design. 
In Section~\ref{sec:single_tone}, we demonstrate observing a single tone from a speaker. 
In Section~\ref{sec:viterbi_wandering}, we observe a wandering frequency signal and analyze it using a hidden Markov model technique (the Viterbi algorithm) from continuous-wave searches. 
In Section~\ref{sec:optical_microphone}, we demonstrate capture and playback of complex audio, such as music and speech.
% removed as there's not space to really say what this means here: to varying degrees of success.
This demonstration of an optical microphone serves as a more general exhibition of signal processing with a range of examples that can be used in the undergraduate laboratory (described in the accompanying Supplementary Material). 
We suggest avenues of future work in Section~\ref{sec:future_work} and draw conclusions in Section~\ref{sec:conclusions}. 
We present the software and scripts used to produce this work in Appendix~\ref{app:code}.


\end{document}
