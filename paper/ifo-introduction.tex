\documentclass[paper-main.tex]{subfiles}

\begin{document}


% general gw intro
In 2015, the first detection of gravitational waves from the merger of two black holes~\cite{GW150914} by the Advanced Laser Interferometer Gravitational-wave Observatory~(aLIGO \citep{AdvancedLIGO:2015}) marked a breakthrough in modern astrophysics and opened up a new window onto the universe. 
% what are gws and how do we detect them...
Gravitational waves, predicted by Albert Einstein in his theory of general relativity~\cite{Einstein:1916}, are disturbances in `the fabric of' spacetime caused by the acceleration of massive objects.
% technically incorrect, but do we need to mention antiaxial asymmetry here? 
Observatories such as aLIGO, aVirgo, GEO600~\cite{GEO600:2010}, and KAGRA~\cite{KAGRA:2013} detect these waves as tiny changes in length: a `stretching and squeezing' of the distance between objects.
In particular, they use the interference of light to measure changes in the lengths of two perpendicular arms, kilometres long.


% continuous gws
To date, the network of gravitational-wave observatories has detected numerous signals from binary black hole~\cite{GW151226,GW170104,GW170814} and binary neutron star~\cite{GW170817,GW170817multi,GW190425} mergers, short-duration transient signals~\cite{GWTC-1:2018,GWOSC:online}. 
However, the network is also searching for continuous gravitational waves: persistent, periodic, near-monochromatic signals, but is as yet unsuccessful~\cite{ScoX1O2Viterbi:2019,RadiometerO1O2:2019,SearchCrossCorrO1:2017,ScoX1ViterbiO1:2017,SearchRadiometerO1:2017,SearchCrossCorrO1:2017,MiddletonEtAlO2LMXBs:2020,SearchTwoSpecS6:2017}.
Predicted sources for such continuous (gravitational) waves include rotating neutron stars in orbit around low mass stellar companions (e.g.\ low mass X-ray binaries, LMXBs). Continuous waves from such sources may be emitted at constant frequency or may wander slowly in frequency over time. 
For example, the rotation frequency of a neutron star in a LMXB can wander over time due to variable accretion of matter (and hence angular momentum transfer) from the stellar companion~\cite{xraybinaries:1997}, and this rotation frequency is predicted to determine the frequency of the continuous waves.
The prime target for continuous wave searches from LMXBs is Scorpius X-1, an LMXB particularly bright in X-rays.


% table-top
Gravitational-wave observatories are extremely complex but are fundamentally based on the classical Michelson interferometer. 
Table-top Michelson interferometers are commonly used in undergraduate laboratory experiments~\cite{UgoliniEtAl:2019} and to demonstrate the science of gravitational wave detection to non-specialist audiences~\cite{ThorLabsIFO,NikhefIFO,AMIGO:online,TTExhibit:2020,LIGOIFOGlue,LIGOIFOMagnets}.
% sound analogy
In this paper, we use a table-top Michelson interferometer as a `toy' gravitational-wave observatory designed to detect sound instead of gravitational waves. That is, instead of detecting changes in the spacetime along the arms, we detect Newtonian motion of the mirrors driven by acoustic vibrations in the air around them. 
The design of the interferometer is detailed in Section~\ref{sec:ifo}.
By playing different sounds through a speaker aimed at the interferometer, capturing the resultant interference pattern, and applying suitable signal processing, we can play-back the original signal and as such turn the interferometer into an `optical microphone', using light to capture sound.
We observe a constant frequency signal in Section~\ref{sec:single_tone}, a wandering frequency signal with application of the Viterbi algorithm (a hidden Markov model technique) from continuous wave searches in Section~\ref{sec:viterbi_wandering}, and more complex audio, such as music and speech, in Section~\ref{sec:optical_microphone}, to varying degrees of success. The latter being less about applying gravitation wave techniques and more about exhibiting general signal processing.
We suggest avenues of future work in Section~\ref{sec:future_work} and draw conclusions in Section~\ref{sec:conclusions}.
% Appendix~\ref{app:code} collects the software and scripts used to produce this work. 


% broader future applications
This demonstration can be used to teach topics ranging from gravitational-wave detection and analysis to electronics, signal processing, and speech enhancement. 
Additionally, it allows students in physics and electrical engineering courses to explore the response of an accessible, yet nontrivial, opto-mechanical system using a hierarchy of data analysis techniques of increasing complexity, up to those used in the search for continuous waves in aLIGO and aVirgo data~\cite{ScoX1O2Viterbi:2019,ScoX1ViterbiO1:2017,SuvorovaEtAl:2017,SuvorovaEtAl:2017}.

Moreover, gravitational-wave detection has led to increased public and media interest in gravitational-wave science. 
Research groups around the world have produced demonstrations and activities to explain their research to general audiences, with efforts ranging from hands-on demonstrations, exhibitions~\cite{TTExhibit:2020,LIGOScienceEducationCentre:online,GravityDiscoveryCentre:online}, and online data analysis tutorials~\cite{GWOSC:online,LOSC:2015}, to phone applications~\cite{LaserLabs:online,SciVR:online,chirp:online}, online games~\cite{BlackHoleHunter:online}, and musical performances~\cite{ArthurJeffesMusic:online,GravitySynthLeonTrimble:online}.%\han{look up Sarah Farmer's project}
This demonstration can be adapted to an effective outreach tool to join those above in explaining gravitational wave science to a non-specialist audience.


\end{document}
