\documentclass[paper-main.tex]{subfiles}


\begin{document}


Continuous wave searches, such as in Refs.~\cite{SuvorovaEtAl:2016,SuvorovaEtAl:2017,SearchTwoSpecS6:2017,SunEtAlSNR:2018,JonesSun:2020}, are often (but not always) performed on long datasets (months to years in duration). The frequency of a continuous wave signals may wander significantly over this timescale, due to stochastic internal processes in the superfluid interior of isolated neutron stars~\cite{MelatosDouglassSimula:2015,Jones:2010} or variable accretion flows onto a neutron star from a stellar companion (e.g. as in LMXBs)~\cite{BildstenTB:1998}. In this context, wandering ``significantly'' means ``across multiple frequency bins'', where the typical width of a frequency bin is the reciprocal of the total observing time~\cite{JKS:1998,ScoX1O2Viterbi:2019}.
The audio analogue to these continuous wave signals is a tone that wanders in frequency; a note that changes pitch. Here, we adapt the data analysis technique used to search for slowly wandering continuous waves signals from spinning neutron stars in aLIGO and aVirgo data~\cite{SuvorovaEtAl:2016,SuvorovaEtAl:2017}. 



One method to search for a wandering signal is to split the time series data into several shorter segments, analyse these individually, and then recombine them under some continuity assumption to coherently analyse the whole set. Most continuous wave searches use detection statistics to quantify the likelihood of a signal being present in the data at each frequency for each time series segment. The detection statistic used depends on the antenna beam pattern of the detector, which varies as the Earth rotates and orbits the Sun, and the type of target~\cite{JKS:1998,SuvorovaEtAl:2017}.
In this work, we take the discrete Fourier transform of each time series segment and use its amplitude at each frequency as a detection statistic. This ensures that we maximize the likelihood of detecting a sinusoidal signal in Gaussian noise, as described in Appendix~\ref{app:sinusoid_likelihood}. The likelihoods (Fourier amplitudes) from each segment are combined to form a grid in time and frequency (a spectrogram). We also normalise the detection statistics for convenience.


We use a hidden Markov model (HMM) to recover the wandering signal from noisy data. To do this, we define the frequency of the signal as a hidden (unknown) state which can undergo transitions at discrete times. We assume that these transitions are Markovian in that the hidden state (i.e.\ frequency) of the system at any time depends solely on its state at the previous time step. The detection statistic (observable) then relates the hidden states of the system to the observed data.
To perform the task of recovering the signal, we use the Viterbi algorithm~\cite{Viterbi:1967} to efficiently find the most probable path through the spectrogram given the sequence of observations.
This method is used in continuous wave searches for a variety of astrophysical targets~\cite{ScoX1O2Viterbi:2019,ScoX1ViterbiO1:2017,MillhouseStrangMelatos:2020,JonesSun:2020,MiddletonEtAlO2LMXBs:2020,PostMergerRemnantSearch:2019,SunEtAlSNR:2018,viterbi_application}.
 

In Section~\ref{sec:viterbi} we review the Viterbi algorithm and its application to this work. 
In Section~\ref{sec:wanderingResults} we present the results for recovery of a wandering audio frequency. 






\subsection{The hidden Markov model and Viterbi algorithm}
\label{sec:viterbi}



The spectrogram, as shown in Fig.~\ref{fig:viterbi}, has $N_t$ time and $N_f$ frequency bins. From here on, we will call these time “steps” instead of time bins, but they should not be confused with the timesteps of the original time series. We label each time and frequency step as $t_i$ and $f_j$ with $i=0,1,2,...N_t$ and $j=0,1,2,...,N_f$. At each state (grid point) the detection statistic (the normalised Fourier amplitude) is notated $F(t_i,f_j)$.


The objective is to find the most likely path through the grid given the observed data and any probabilistic constraints on how the frequency of the signal can wander between time steps $t_i$ to $t_{i+1}$. In continuous wave searches, a physical model of the target informs how far the frequency of the signal can wander over each time step. This information is encoded in the transition probability matrix $A(f_k,f_m)$, which describes the probability of system transitioning from state $f_i$ at $t_i$ to a state $f_{j}$ at $t_{i+1}$. Here, we allow the frequency of the signal to either (a) stay in the same bin; (b) move up by a single frequency bin; or (c) move down by a single frequency bin at each time step. We assign these three transitions equal probability, i.e.\ $A(f_k,f_m)=1/3$ for $k=m+1,m,m-1$ and $A(f_k,f_m)=0$ otherwise.



We define the probability of the system having frequency $f_j$ at the initial time $t_0$ to be equal to the (normalised) detection statistic at that state (i.e.\  ${\rm Pr}[f(t_0) = f_j] = F[t_0,f_j]$).
The probability of the system having frequency $f_j$ at any later time $t_{i+1}$ is then defined recursively by
\begin{eqnarray}
{\rm Pr}[f(t_{i+1})=f_j] =~& F[t_{i+1},f(t_{i+1})] \nonumber \\
                     &\times A[f(t_{i+1}),f(t_i)]  \nonumber \\
                     &\times {\rm Pr}[f(t_i)].
\label{eqn:recursiveViterbi}
\end{eqnarray}
The sequence $f(t_0),f(t_1),\dots,f(t_{N_t})$ that maximizes ${\rm Pr}[f(t_{N_t}) = f_j]$ is the optimal Viterbi path terminating in the frequency bin $f_j$. 
We can the maximize the latter quantity over $0 \leq j \leq N_f$ to find the optimal Viterbi path overall, i.e. terminating in any frequency bin. 


The Viterbi algorithm provides a computationally efficient method for finding the optimal path, at every time step all but $N_f$ possible paths are eliminated (see also Ref.~\cite{ScoX1ViterbiO1:2017}). Here we describe the algorithm in pseudocode while referring to the schematic in Fig.~\ref{fig:viterbi}, where the circles indicate the states in the spectrogram with the lighter colours corresponding to a higher likelihood. The implementation used in this work is available online (see Appendix~\ref{app:code}).
\begin{enumerate}
\item Starting at time $t_1$, each $f_j$ state can originate from three prior states at time $t_0$ (except for $f_0$ and $f_{N_f}$ which only have two). The paths between these states are indicated by the lines in Fig.~\ref{fig:viterbi}. At each $f_j$ state, we select the path with the highest $A[f(t_0),f(t_1)] {\rm Pr}[f(t_0)]$ as the most probable path. These choices are highlighted using the black lines in Fig.~\ref{fig:viterbi} while the grey lines show the rejected paths. For example, the most probable connection to the element labelled (a) is the one directly behind it (i.e., $f_4$). Therefore, this path is selected as the best path from $t_0$ to $t_1$. For backtracking at the end, the index of the most probable connection along with the current cost of the best path is stored for each node.

\item Moving to the third time step, $t_2$, again, we select the path which maximises the recursive quantity in Eqn.~\ref{eqn:recursiveViterbi} for each $f_j$. These are again shown by the solid black lines between the nodes at $t_1$ and $t_2$ in Fig.~\ref{fig:viterbi}.

\item Step 2 is repeated until the end of the grid ($t=t_{N_t}$) is reached with only the best paths being stored at each iteration. 

\item Each state $f_j$ at $t=t_{N_t}$ has some value ${\rm Pr}[f(t_{N_t})=f_j]$ (Eqn.~\ref{eqn:recursiveViterbi}), which represents the probability of the most likely path that ends in state $f_j$ at $t_{N_t}$. The Viterbi algorithm then selects the terminating frequency bin $f(t_{N_t})$ with the highest probability, marked by (b) in Fig.~\ref{fig:viterbi}.

\item The Viterbi path (the overall best path) is then found by backtracking along the stored best connections at each time step (see also Appendix~\ref{app:viterbi}). In Fig.~\ref{fig:viterbi}, we see that the path ending at (b) started at (c) and is highlighted in orange in Fig.~\ref{fig:viterbi}.
\end{enumerate}

In the following section, we use the Viterbi algorithm to recover a slowly wandering signal.


% The Viterbi algorithm is an example of a dynamical programming algorithm, where a computation can be broken down into a series of sub-computations.
% The Viterbi path for a time series sequences spanning $t_0$ to $t_{N_t}$ contains the Viterbi path for sub-sequences of that time series. 
% The algorithm is computationally efficient, at every time step all but $N_f$ possible paths are eliminated (see also Ref.~\cite{ScoX1ViterbiO1:2017}).

\begin{figure}
\includegraphics[width=0.49\textwidth]{figures/viterbiDiagram.pdf}
\caption{\label{fig:viterbi}
A schematic diagram of the Viterbi algorithm. 
Each box represents an element in the time-frequency grid starting from the instant $t_0$ on the left and ending at $t_{N_t}$ on the right. 
Vertically, the grid starts from frequency $f_0$ at the bottom and ending at $f_{N_f}$ at the top. 
The colour of each circle indicates the likelihood that a signal is present in the box, where a lighter colour corresponds to higher likelihood. 
AT each time step, the paths can move up one $f$ step, move down one $f$ step, or stay at the same $f$ step (shown by lines in the diagram). 
Each circles has three paths leading to it indicated by the lines. 
The best path to each circle at each $t_i$ is highlighted in black. 
Some routes through the grid are found to be dead ends, such as the path ending at $t_1$ marked (a). 
At $t=t_{N_t}$ the algorithm chooses the terminating frequency circle which has the highest value given by Eqn.~\ref{eqn:recursiveViterbi}, marked (b). 
The Viterbi path is the path leading to this circle, highlighted in orange from (b) to the start at (c). 
}
\end{figure}



\subsection{Sample output}
\label{sec:wanderingResults}

\begin{figure*}
	\includegraphics[width=\textwidth]{figures/expt_overlay_2_viterbi_test_webcam.pdf}
	\caption{\label{fig:viterbi_overlay}
Recovery of a wandering tone. 
The spectrogram shows the observed frequency amplitude at each time-frequency step. 
The overlaid pink-dot and white-cross markers show the injected signal and recovered Viterbi path respectively. 
On the left, before $\sim 15\,{\rm s}$, the signal changes frequency too quickly for the Viterbi algorithm to recover. 
At $150\,{\rm s}$ the data appears anomalous, which may be due to some background noise. }
\end{figure*}
 
% throughout this section we are using present tense, should this be changed to past tense?

A sinusoidal signal with a slowly decaying amplitude is injected into the interferometer using the speaker in the setup shown in Fig~\ref{fig:ifo_schematic_webcam} to mimic a slowly wandering signal. The particular choice of signal is arbitrary as any injected path would have worked, as long as it changed slowly enough for the search parameters of the algorithm. The output of the interferometer is recorded via a webcam as in Section~\ref{sec:single_tone}. We then test the Viterbi algorithm’s ability to recover the wandering signal.


The results are shown in Fig.~\ref{fig:viterbi_overlay}, where the heatmap shows the spectrogram of the observed signal. The overlaid pink dots show the injected signal while the white crosses show the recovered Viterbi path (the most likely path through the data according to the Viterbi algorithm). The recovered Viterbi path is within one frequency bin ($\approx 0.3\,{\rm Hz}$) of the injected signal for $94\%$ of the time. We also compute the root mean square (RMS) fractional error $E_{\rm rms}$ along the path which is defined as 
\begin{equation}
E_{\rm rms} = \left[ \frac{1}{N_t+1} \sum_{i=0}^{N_t} \frac{(I_i - R_i)^2}{{I_i}^2} \right] ^{1/2}
\end{equation}
where $I_i$ and $R_i$ are the injected and recovered (Viterbi) frequency paths. Computing this for the result shown in Fig.~\ref{fig:viterbi_overlay}, we find that $E_{\rm rms} = 0.082$, which indicates fair but not total recovery of the injected signal.


This may be explained by two anomalies in the recovered path. Initially, the injected signal wanders by more than one frequency bin at each time step (i.e.\ faster than the algorithm is allowed to), thus leading to a discrepancy between the injected and recovered paths for $t\lesssim 10\,{\rm s}$. One may be tempted to increase the allowed frequency wander in the algorithm, however this leads to an overall decrease in the above statistics, as the algorithm is prone to jump briefly to nearby spots of noise. There is also an anomaly at $150\,{\rm s}$, which is likely due to a local disturbance, e.g. someone walked past the interferometer. The Viterbi algorithm is somewhat robust against such disturbances, qualitatively recovering the injected path on the other side, as shown in Fig.~\ref{fig:viterbi_overlay}. 



\end{document}
