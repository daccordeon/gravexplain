\documentclass[a4paper, 10pt]{letter}

\usepackage{fullpage}


\begin{document}


Dear Joseph, 

Thank you for the comments and feedback. 
We attach a new version of the manuscript.

We completely understand about the reference list and we have moved the majority of them into a ``Further Reading'' section in the Supplementary Material. 

You also asked about whether we have had any experience using this apparatus in the classroom. Unfortunately we have not and we are unsure when this will be possible as we are in and out of lockdown in Melbourne. Before COVID, we used the interferometer at several science fairs and schools, however it did not have the audio set-up at that stage, so I think that experience is not directly relevant to this work. 

We have also made changes based on the annotated PDF from you and below answer the annotations where clarifications, etc. were needed. 

Thank you again for the feedback. 

Best wishes, \\
Hannah (for the co-authors)
 
\begin{itemize}
\item p.4 (Figure 4 comments):  I think it is obvious that a single FT of a frequency-varying signal cannot capture the signal, so I don't see the need for Figure 4. 
\begin{itemize}
\item We are happy to remove figure 4. We added it only after the initial review in response to one of the suggestions to add a comparison of what would be detected without using Viterbi. We agree, however, that it does not add much to the explanation. We have kept a shortened version of the highlighted text to explain why we cannot use the Fourier transform in this case and we have removed the figure. 
\end{itemize}

\item p.4 Section IV A: I appreciate your desire to be all-inclusive, but is the highlighted text necessary? It is not directly related to the topic of your paper. 
\begin{itemize}
\item This expanded literature review was one of the requests from Referee 3. We agree that they are tangentially related to the work presented, so we have instead moved them to the Further Reading section in the Supplementary Material. 
\end{itemize}

\item p.5 Figure 5: This won't show up in B/W. How about indicating the probability by different sizes of the circles? 
\begin{itemize}
\item We found it very difficult to make a tidy figure using different sized circles. Instead, we now use a grey colour scheme, which looks clear and should be fine in black and white print. 
\end{itemize}

\item p.5, left column, para 1: Just to make sure I understand: the color of each circle is what you observe?
\begin{itemize}
\item Yes, that's correct. We have added some clarification to the paragraph.
\end{itemize}

\item p.5, right column, para 2: The first two sentences seem to contradict each other. Can you clarify? 
\begin{itemize}
\item The first sentence refers to our prior beliefs before we begin any analysis or look at the data. Without any prior knowledge, we believe that the signal is equally likely to have started in any frequency bin within the range considered. The second sentence is the first stage of the algorithm once we have begun the analysis and looked at the data. We have added clarification to the text to explain this. 
\end{itemize}

\item p.5, right column, para 1: What does ``marginalization'' mean? Do you mean no hard barriers?
\item This statement was added in response to referee 3, who asked about marginalisation and spotted an error in this equation in the initial version of the manuscript. If we have a probability distribution in multiple dimensions, but we only want to consider the probability distribution of a single dimension, we would sum over (or marginalize) all of the other unwanted dimensions. This is not what we are doing in this instance. The equation corresponds to a single probability related to a unique path. We added this statement about marginalization to ensure no confusion given the referee's question, however we believe the meaning is still maintained without it. We have removed the first part of this sentence, keeping only the second part. 

\item p.5, right column, list item 1: ``node'' meaning one of the circles in Fig. 5? 
\begin{itemize}
\item Yes, the nodes does mean the circles. Thank you, this was confusing and we've made the language more consistent now and referred to the circles when we first mention nodes.
\end{itemize}

\item p.5, right column, list item 3: In the example of Fig. 5, since A=1/3 or 0 for each transition, it looks like the selected path just links the states with the highest Fourier amplitude. Is it always that simple?
\begin{itemize}
\item That is a good point and it is not always so simple. In our Figure 5, we have made an example where the signal is particularly loud and can easily be seen by eye, but it is not always the case. In continuous-wave searches we expect that the signal cannot be seen by eye. It might also be the case that there are loud noise fluctuations in time-frequency bins. In this case they may not be selected by the Viterbi algorithm if the overall value of equation 5 is higher for the signal path than the paths containing the noise. 
\end{itemize}

\item p.6, left column, after itemized list: delete `expected to be' 
\begin{itemize}
\item We believe it is important to keep ``expected to be''. The reason is that continuous waves have not yet been detected and although of course we believe they are worth searching for, a more pessimistic person might argue that the signals may not be there at all! 
\end{itemize}

\item p.6, left column: How does the assumption of white noise get inserted into the VA? How does this assumption affect the results?
\begin{itemize}
\item The white noise assumption is implicit in that we do not build any knowledge of coloured noise into the analysis. This statement was added in response to referee 3's observation that the noise in our data is not white and they asked how or if we take account of this. We do not take account of non-white noise in our analysis (which is also consistent with LIGO analyses which assume white noise even though LIGO noise is not white). We also note that the narrow frequency range we consider is less impacted by the increase in noise at low frequency. We have added some details to clarify this in the text. 
\end{itemize}

\item p.6, right column: If you eliminate the start-up points from the calculation, does E change significantly?
\begin{itemize}
\item We have also computed the $E_{\rm rms}$ value when the first four time bins are ignored. We find a slight improvement in recovery and have added a sentence about this to the manuscript to our results discussion. 
\end{itemize}

\item p.7, Fig. 6: It looks like the frequency bins are 0.3 Hz wide and 3s long. Can you estimate the S/N ratio in this data set to illustrate the power of the VA?
\begin{itemize}
\item We can only estimate S/N for single time blocks of data, however we do not believe that a representative S/N can be estimated for the entire data set. We have clarified this in the text were we refer to the signal to noise ratio of real continuous-wave signals in Section IV.C.
\end{itemize}

\item p.7, right column, Section V, para 2: Just to be clear, what have you changed from the apparatus in earlier sections (if anything), other than the photodiode described below? It reads like you have changed something.
\begin{itemize}
\item The only thing we have changed is replacing the webcam with the photodiode so that the apparatus can be used to capture higher frequency speech and music. In this section we use the interferometer to capture more complicated signals, so we draw comparisons to laser microphones used in other fields. We have added a sentence to this paragraph to clarify the apparatus changes. 
\end{itemize}

\item p.8, left column: This paragraph adds very little new information to the topic of Section V, and interrupts the flow of the paper. I recommend shortening it. 
\begin{itemize}
\item Agreed. This paragraph was added in response to reviewer 3's request for us to mention other methods of noise reduction from the gravitational-wave literature. This seemed like the most appropriate place for it to go, however we agree that it is not closely related to the work presented here. We have shortened this discussion to a single sentence and moved the literature review to the Further Reading section in the Supplementary Material. 
\end{itemize}

\item p.9, Figure 8: It's hard to judge the effectiveness of the filtering on speech intelligibility from the graphs. Here's an idea: we could attach a direct link to an audio clip for each row, perhaps a 10 second recitation of a familiar phrase. It would be available to online readers via s simple click. This would enhance the discussion significantly.
\begin{itemize}
\item This sounds a great idea! We have collated the four sounds files and attached them in the email. Please let us know if you need any further details from us. 
\end{itemize}

\item p.10, left column, para 2: This seems like an unlikely approach. Why not simply increase the sensitivity, as described in the following sentence, and rotate the apparatus relative to the source?
\begin{itemize}
\item We have clarified this in the text. The gain in having data from two interferometer is that directional information can be estimated based on the time delay between the signal arriving at each interferometer, in a similar way to how we are able to tel the direction of a sound with our ears. This is also how gravitational-wave detectors estimate the direction of compact binary mergers based on the time-delay of the signal reaching two or more detectors. 
\end{itemize}

\item p.10, left column, para 2: Again, this seems like an unlikely strategy, given the sensitivity of the apparatus to mechanical noise. 
\begin{itemize}
\item We agree, this is likely to be a fair assessment about the noise of the system. We have removed the part about moving the interferometer. 
\end{itemize}

\item p.10, conclusions comments
\begin{itemize}
\item We have removed the repetition and shorted the conclusion. 
\end{itemize}

\item p.11, appendix B: ``The result you derive is a dc signal, which carries no information, plus a frequency-doubled component for small delta-d. But shouldn't there also be a component at the initial frequency, akin to frequency modulation? Otherwise, Fig. 3 cannot be explained.''
\begin{itemize}
\item Thank you for the careful reading and we agree completely - this was an error on our part in the derivation. We have decided to remove this appendix. 
\end{itemize}

\item p/11, appendix C: ``This needs more explanation. I certainly don't understand it.''
\begin{itemize}
\item We have expanded the text to clarify the explanation. 
\end{itemize}

\item Other appendix changes
\begin{itemize}
\item In order to reduce references, we have moved the software references in Appendix A and the whole of Appendix E into the Supplementary Material. 
\end{itemize}

\end{itemize}

\end{document}
