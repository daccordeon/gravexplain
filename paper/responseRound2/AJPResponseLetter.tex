\documentclass[a4paper, 10pt]{letter}

\usepackage{fullpage}


\begin{document}


Dear Joseph, 

Thank you for the comments and feedback. 
We attach a new version of the manuscript.

We completely understand about the reference list and we have moved the majority of them into a ``Further Reading'' section in the Supplementary Material. 

You also asked about whether we have had any experience using this apparatus in the classroom. Unfortuntately we have not and we are unsure when this will be possible as we are in and out of lockdown in Melbourne. Before COVID, we used the interferometer at several science fairs and schools, however it did not have the audio set-up at that stage, so I think that experience is not directly relevant to this work. 

We have also made changes based on the annotated PDF from you and below answer the annotations where clarifications, etc. were needed. 

Thank you again for the feedback. 

Best wishes, \\
Hannah (for the co-authors)
 
\begin{itemize}
\item p4 Figure 4 
\begin{itemize}
\item We are happy to remove figure 4. We added it only after the inital review in response to one suggestion of adding a comparison of what would be detected without using Viterbi, but I agree that it does not add much to the explanation. We have kept a shortened version of the higlighted text to explain and we have removed the figure. 
\end{itemize}

\item p4 Section IV A
\begin{itemize}
\item The expanded literature review here was one of the requests from Referee 3. We agree that they are tangentally related to the work presented, so we have instead moved them to the Further Reading section in the Supplementary Material. 
\end{itemize}

\item p5 Figure 5 [about Figure colour scheme]
\begin{itemize}
\item We found it very difficult to make a tidy figure using different sized circles. Instead, we have tried a gray color scheme, which looks okay and should be fine in black and white print. 
\end{itemize}

\item p5, right column, para 2, line 1
\begin{itemize}
\item I see what you mean. The first sentence refers to our ``prior beliefs'' before we look at any data. The second sentence is about the first step of the analysis once we have the data.  
[to do]
\end{itemize}

\item p5, right column, list item 1
\begin{itemize}
\item Yes, the nodes does mean the circles. Thank you, this was confusing and we've made the language more consistent now and refered to the circles when we first mention nodes.
\end{itemize}

\item p5, right column, list item 3: ``In the exampler of Fig. 5, since A=1/3 or 0 for each transition, it looks like the selected path just links the states with the highest Fourier amplitude. Is it always that simple?''
\begin{itemize}
\item That is a good point and it is not always so simple. In our Figure 5, we have made an example where the signal is particularly loud and can easily be seen by eye, but it is not always the case. In continuous-wave searches we expect that the signal cannot be seen by eye. It might also be the case that a loud single frequency-time bin is not part of Viterbi path recovered. [what to change?]
\end{itemize}

\item p6, left column, after itemized list: ``delete `expeced to be' ''
\begin{itemize}
\item I think we should keep ``expected to be''. The reason is that continuous waves have not yet been detected and although of course we believe they are worth searching for, a more pessimistic person might argue that the signals might not be there at all! 
\end{itemize}

\item p6, left column: ``How does the assumption of white noise get inserted into the VA? How does this assumption affect the results?'' 
\begin{itemize}
\item [to do]
\end{itemize}

\item p6 right column: ``If you elimintate the start-up points from the calculation, does E change significantly?''
\begin{itemize}
\item We have also computed the $E_{\rm rms}$ value when the first four time bins are ignored. We find a slight impovement in recovery and have added a sentence about this to the manuscript at the end of this paragraph. 
\end{itemize}

\item p7, Fig. 6: ``It looks like the frequency bins are 0.3 Hz wide and 3s long. Can you estimate the S/N ratio in this data set to illustrate the power of the VA?''
\begin{itemize}
\item We could estimate S/N for single time blocks of data, however we do not believe that a representative S/N can be estimated for the entire data set. 
\end{itemize}

\item p7, right column, Section V, para 2. ``Just to be clear, what have you changed from the apparatus in earlier sections (if anything), other than the photodiode described below? It reads like you have changed something.''
\begin{itemize}
\item The only thing we have changed is replacing the webcam with the photodiode. As we now using the interferometer to capture more complicated speech and music, we draw comparisons to laser microphones used in other fields.  [to do ]
\end{itemize}

\item p9 Figure 8: ``It's hard to judge the effectiveness of the filtering on speech intelligibility from the graphs. Here's an idea: we could attach a direct link to an audio clip for each row, perhaps a 10 second recitation of a familiar phrase. It would be available to online readers via s simple click. his ould enhance the discuation sigificantly.''
\begin{itemize}
\item [to do ]
\end{itemize}

\item p10 left column, para 2
\begin{itemize}
\item I've tried to clarify the extension involving two interferometers. It would enable us to estimate the direction of the source using the relative time of arrival of the signal at each interferometer. This would help to demonstrate how gravitational-wave detectors constrain the direction of transient gravitational wave signals. 
\end{itemize}

\item p11, appendix B: ``The result you derive is a dc signal, which carries no informatoin, plus a frequency-doubled component for small delta-d. But shouldn't there also be a component at the initial frequency, akin to frequency modulation? Otherwise, Fig. 3 cannot be explained.''
\begin{itemize}
\item Thank you very much for the careful reading and we agree completely - this was an error on our part in the derivation. We have decided to remove this appendix. 
\end{itemize}

\item p11, appendix C: ``This needs more explanation. I certainly don't understand it.''
\begin{itemize}
\item [to do]
\end{itemize}

\end{itemize}

\end{document}
