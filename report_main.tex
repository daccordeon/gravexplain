% GravExplain
\documentclass[prb,preprint]{revtex4-1} 
% \documentclass[prb,preprint,letterpaper,noeprint,longbibliography,nodoi,footinbib]{revtex4-1} 

% Note that AJP uses the same style as Phys. Rev. B (prb).
% The AIP Style Manual\cite{AIPstylemanual} is an indispensable reference on good physics writing, covering everything from planning and organization to standard spellings and abbreviations.
% Most important of all, please familiarize yourself with the AJP Statement of Editorial Policy,\cite{editorsite} which describes the types of manuscripts that AJP publishes and the audience for which AJP authors are expected to write.
% We look forward to receiving your submission to AJP.
\usepackage[utf8]{inputenc}
\usepackage{amsmath,amssymb,amsthm}
\usepackage{amsfonts}
\usepackage{graphicx}
\usepackage{float}
\usepackage{mathtools}
\usepackage[usenames,dvipsnames]{xcolor}
\usepackage{hyperref}

\bibliographystyle{apsrev4-2}
% \setlength{\parindent}{0pt}

\newcommand{\jam}{\textcolor{magenta}}


\begin{document}

\title{GravExplain: Continuous gravitational wave data analysis in the undergraduate laboratory}

\author{James W. Gardner}
\email{u6069809@anu.edu.au}
\affiliation{Research School of Physics, Australian National University, 60 Mills Rd, Acton ACT 2601 Australia}

\author{Hannah Middleton}
\email{hannah.middleton@unimelb.edu.au}
\author{Andrew Melatos}
\email{amelatos@unimelb.edu.au}
\affiliation{School of Physics, University of Melbourne, Parkville, VIC, 3010, Australia}
\affiliation{OzGrav-Melbourne, Australian Research Council Centre of Excellence for Gravitational Wave Discovery}

% Summer 2019/202
\date{\today}

% AJP requires an abstract for all regular article submissions.
\begin{abstract}
GravExplain
	
\end{abstract}

\maketitle

\section{Introduction}
% see paper?


\subsection{Interferometry}
% path difference, virtual thin film interference


\section{Method}

\subsection{Data capture: webcam}
% how to analyse video, python OpenCV2

\subsection{Advanced data capture: photodiode}
% leave circuit design to an appendix


\section{Results and analysis}
% combined/interwoven results, analysis, and discussion

\subsection{Tracking tones: the Viterbi algorithm}
% explaination and successful application
\jam{should we introduce viterbi in the introduction, probably?}

\subsection{Hearing harmonies and chords}
% the best we can do right now
% write about removing mains noise
\jam{this is as far as we go right now}

\subsection{Reconstructing music and voice}
% signal processing required
% may well be a ``future work'' section at the end
\jam{! purely hypothetical right now !}


\section{Conclusions}
% we are able to successfully ...


\appendix
\section{}


\begin{acknowledgments}
We gratefully acknowledge Jude Prezens, Alex Tolotchkoc, and Blake Molyneux for their technical advice and generous assistance.

The authors are grateful to Deeksha Beniwal, Sebastian Ng, and Craig Ingram for their advice and work in designing the interferometer. 

J.~Gardner also acknowledges the ANU PhB program for providing travel funding during the project.

This research is supported by the Australian Research Council Centre of Excellence for Gravitational Wave Discovery (OzGrav) (project number CE170100004) and the Institute of Physics International Member Grant.


\end{acknowledgments}


\bibliographystyle{myunsrt}
\bibliography{ifoDemoBib}

\end{document}
